\documentclass[12pt,a4paper]{article}

% Paquetes
\usepackage[utf8]{inputenc}
\usepackage[T1]{fontenc}
\usepackage[spanish]{babel}
\usepackage{graphicx}
\usepackage{geometry}
\usepackage{booktabs}
\usepackage{tabularx}
\usepackage{caption}
\usepackage{hyperref}
\usepackage{setspace}
\usepackage{apacite}
\usepackage{float}
\usepackage{placeins}

% Márgenes
\geometry{top=2.5cm, bottom=2.5cm, left=3cm, right=3cm}

% Datos
\title{Práctica de Laboratorio 3: Pruebas de Rendimiento y Planificación de Procesos}
\author{Wílmer E. León et al.}
\date{Fecha: \today}

\begin{document}
\maketitle
\tableofcontents
\newpage
\onehalfspacing

\section{Introducción}
En esta entrega se documentan pruebas adicionales para evaluar el rendimiento del balanceo de carga configurado previamente y una profundización en algoritmos de planificación de procesos en sistemas operativos.

\section{Parte 3: Pruebas Adicionales y Evaluación del Rendimiento}
\subsection{Paso 1: Preparación de Pruebas}
Se propone el uso de herramientas como Apache Benchmark (\texttt{ab}) o \texttt{siege} para generar tráfico HTTP hacia el balanceador de carga. Diseñe al menos dos escenarios: tráfico ligero (p. ej. 100 peticiones) y tráfico pesado (p. ej. 10.000 peticiones concurrentes con X concurrencia).

\subsection{Paso 2: Ejecución de Pruebas}
Ejecute las pruebas desde una máquina cliente. A continuación se muestra un ejemplo con \texttt{ab} (esto también está en \texttt{run\_tests.ps1}):
\begin{verbatim}
ab -n 1000 -c 50 http://192.168.2.9/
\end{verbatim}

Incluya capturas de las métricas relevantes y logs. A modo de ejemplo se utiliza la imagen del test previo:
\begin{figure}[H]
  \centering
  \includegraphics[width=0.8\textwidth]{capturas/nginx-test.png}
  \caption{Ejecución de pruebas y salida de ejemplo.}
  \label{fig:nginx-test}
\end{figure}

\subsection{Paso 3: Registro de métricas}
Registre métricas por prueba: tiempo de respuesta promedio (ms), requests por segundo, tasa de error, CPU y memoria de backends. Use tablas para resumir resultados.

\begin{table}[H]
\centering
\caption{Resumen de métricas ejemplo}
\begin{tabularx}{0.9\textwidth}{lXXXX}
\toprule
Escenario & Requests/s & Tiempo medio (ms) & CPU promedio (\%) & Errores \\
\midrule
Ligero & 1200 & 85 & 12 & 0 \\
Pesado & 420 & 260 & 75 & 2 \\
\bottomrule
\end{tabularx}
\end{table}

\subsection{Paso 4: Análisis de Resultados}
Analice variaciones entre escenarios y busque cuellos de botella (ej.: CPU alta en backends, saturación de red, tiempos de espera altos). Proponga mitigaciones: añadir réplicas backend, ajustar timeouts, balanceo por least-connections.

\section{Profundización: Planificación de Procesos en Sistemas Operativos}
\subsection{Paso 5: Algoritmos seleccionados}
Se analizarán los algoritmos: FIFO (FCFS), SJF (Shortest Job First) y Round Robin (RR).

\subsection{FIFO (First-In-First-Out)}
\textbf{Funcionamiento:} Los procesos se atienden en orden de llegada.
\textbf{Criterio de selección:} orden de llegada.\\
\textbf{Ventajas:} sencillo, justo para llegada secuencial.\\
\textbf{Desventajas:} puede causar tiempos de espera largos (convoy effect).\\
\textbf{Casos de uso:} entornos con cargas homogéneas y poca concurrencia.

\subsection{SJF (Shortest Job First)}
\textbf{Funcionamiento:} se selecciona el proceso con menor tiempo de ejecución esperado.
\textbf{Criterio de selección:} duración estimada del trabajo.\\
\textbf{Ventajas:} minimiza tiempo de espera promedio.\\
\textbf{Desventajas:} requiere estimaciones; puede producir inanición de trabajos largos.\\
\textbf{Casos de uso:} sistemas por lotes donde se conocen tiempos de ejecución.

\subsection{Round Robin (RR)}
\textbf{Funcionamiento:} cada proceso recibe un quantum de CPU por turno.
\textbf{Criterio de selección:} tiempo circular con quantum fijo.\\
\textbf{Ventajas:} buena respuesta interactiva, equidad.\\
\textbf{Desventajas:} elección de quantum crítica; context switches frecuentes si quantum pequeño.\\
\textbf{Casos de uso:} sistemas interactivos y multiusuario.

\subsection{Comparación}
\begin{table}[H]
\centering
\caption{Comparación simplificada de algoritmos}
\begin{tabularx}{0.95\textwidth}{lXXXX}
\toprule
Algoritmo & Tiempo respuesta & Tiempo espera & Eficiencia & Uso recomendable \\
\midrule
FIFO & Alto en promedio & Alto & Medio & Trabajos secuenciales \\
SJF & Bajo & Bajo & Alto & Batch con estimaciones \\
RR & Medio-Bajo & Medio & Variable & Sistemas interactivos \\
\bottomrule
\end{tabularx}
\end{table}

\section{Conclusiones}
Resuma hallazgos sobre el rendimiento del balanceador y las recomendaciones sobre planificación de procesos.

\FloatBarrier
\section{Referencias}
% Incluir todas las entradas de referencias3.bib aunque no se citen explícitamente
\nocite{*}
\bibliographystyle{apacite}
\bibliography{referencias3}

\end{document}
